\documentclass{standalone}

\usepackage[latin1]{inputenc}
\usepackage{tikz}
\usetikzlibrary{shapes,arrows}
\usetikzlibrary{shapes}
\usetikzlibrary{shapes}
\usetikzlibrary{decorations.pathmorphing}
\usetikzlibrary{decorations.text}
%%%<
\usepackage{verbatim}
\usepackage[active,tightpage]{preview}
\PreviewEnvironment{tikzpicture}
\setlength\PreviewBorder{5pt}%
%%%>
% %SMART DIAGRAM START
\usepackage{lmodern}
\usepackage{tikz}

\usetikzlibrary{calc,shadows} 
\makeatletter 
\@namedef{color@1}{red!30}
\@namedef{color@2}{cyan!30}   
\@namedef{color@3}{blue!30} 
\@namedef{color@4}{green!30}  
\@namedef{color@5}{magenta!30} 
\@namedef{color@6}{yellow!30}    
\@namedef{color@7}{orange!30}    
\@namedef{color@8}{violet!30}   

\newcommand{\smartart}[1]

% %END SMART DIAGRAM

% 
\usepackage{amsmath,bm,times}
\newcommand{\mx}[1]{\mathbf{\bm{#1}}} % Matrix command
\newcommand{\vc}[1]{\mathbf{\bm{#1}}} % Vector command


%

\begin{document}
\pagestyle{empty}
% We need layers to draw the block diagram
\pgfdeclarelayer{background}
\pgfdeclarelayer{foreground}

\pgfsetlayers{background,main,foreground}

\usetikzlibrary{positioning,shapes,shadows,arrows}
\usetikzlibrary{positioning, arrows,shadows}

\tikzset{
dropshadowfade/.style={drop shadow={shadow xshift=0.2em,
shadow yshift=-0.2em, opacity=0.5
},
preaction={drop shadow ={shadow xshift=0.30em, shadow
yshift=-0.30em, opacity=.3} ,
preaction={
drop shadow={opacity=.1,
shadow xshift=0.4em, shadow yshift=-0.4em}}},
},
state/.style={draw, rounded corners, inner sep=1em, fill=blue!5,
dropshadowfade,
minimum height=1.5em,
minimum width=10em },
trans/.style={->,draw,rounded corners=1em,thick,>=stealth'},
transloopright/.style={out=15, in=345, looseness=3},
transloopabove/.style={out=105, in=75, looseness=3},
}

\begin{tikzpicture}[node distance = 4cm, auto]
%\node[rectangle,
%    rounded corners,
%     thick,
%     draw=gray,
%     top color= white,
%     bottom color=\col,
%     drop shadow,
%     text width=1.75cm,
%     minimum width=2cm,
%     minimum height=1cm,
%     font=\small] (satellite\xi) at (\angle:2.75cm) {\gritem };
%}%

% Place nodes
\node [state,fill =blue!40,xshift=2.5em] (jobtracker) {Job Tracker};
%\fit[state, fill=blue!10]{}

\begin{pgfonlayer}{background}
\node [state,minimum height=5.5em,label={[shift={(1.3,-2)}]Job Tracker},minimum width=12.5em,fill=blue!10, fill opacity=1,xshift=2.5em] (jtn){};
\end{pgfonlayer}


\node [state, left of=jobtracker,xshift =-11em,fill =blue!40] (jobclient) {JobClient};
%\filldraw[fill=green!20!white, draw=green!40!black]jobclient;

\node [state, left of=jobclient,xshift =-6.3em,fill=yellow!40] (mapreduce) {MapReduce Program};

\begin{pgfonlayer}{background}
\node  [state,left of=jobtracker,minimum height=10.5em,minimum width=34em,fill=blue!10,fill opacity=1,xshift=-20.5em,label={[shift={(-1,-3.7)},black]JobClient Node}](jj){};\end{pgfonlayer}

\begin{pgfonlayer}{background}
\node[state,left of=jobtracker,minimum height=5.5em,minimum width=30.5em,fill=white,fill opacity=1,xshift=-20em,label={[shift={(0,-2)},black]Client JVM}] (IMU){};\end{pgfonlayer}


\node[cloud, cloud puffs=15.7, minimum width=3cm, draw, below of=jobclient,yshift=-5em,xshift=-4em,fill=blue!20] (cloud)  {Shared HDFS};

\node [state,below of=jobtracker,yshift =-5em,fill =blue!40] (tasktracker) {TaskTracker};
 

%\node  [state,below of=child,minimum height=12.5em,minimum width=13.5em,fill=yellow, fill opacity=0.2,yshift=8em] (ttn){};
\begin{pgfonlayer}{background}
\node  [state,below of=tasktracker,minimum height=21.5em,minimum width=15.5em,fill=blue!10,fill opacity=1,yshift=3em,label={[shift={(0,-7.6)},black]TaskTracker Node}] (ttn){};
\end{pgfonlayer}
\begin{pgfonlayer}{background}

\node  [state,below of=tasktracker,minimum height=12.5em,minimum width=13.5em,fill=white,fill opacity=1,yshift=0em,label={[shift={(0,-4.4)},black]Child JVM}] (cjvm){};
\end{pgfonlayer}


\node [state,below of=tasktracker,yshift =4em,fill =blue!40] (child) {Child};

\node [state,below of=child,yshift =4em,fill=yellow!40] (maptask) {MapTask or ReduceTask};


\path [trans,dashed,black] (mapreduce) edge [] node[auto,black,fill =white,xshift=0em,yshift=0em] {1:run job} (jobclient);

\path [dashed,black] ([yshift=1em] jobclient.east) edge [trans] node[auto,black,fill=white,xshift=1em] {2:get new job ID} ([yshift=1em]jobtracker.west);

\path [dashed,black] ([xshift=-3.7em] jobclient.south) edge [trans] node[auto,black,fill =white,xshift=0em,yshift=0em] {3:copy job resources} (cloud);

\path [dashed,black] ([yshift=-1em] jobclient.east) edge [trans] node[auto,black,fill =white,xshift=0.6em,yshift=-1.7em] {4: submit job} ([yshift=-1em]jobtracker.west);

\path [trans,dashed,black,->] ([xshift=-3em] jobtracker.north) -- ++(0,1) -| node[auto,black,fill =white,xshift=-7em,yshift=1em] {5: initialize job} ([xshift=4em]jobtracker.north);

\path [dashed,black] ([xshift=-3em] jobtracker.south) edge [trans] node[auto,black,fill =white,xshift=0em,yshift=0em] {6: retrieve input split} (cloud);

\path [dashed,black] ([xshift=0em] tasktracker.north) edge [trans] node[auto,black,fill =white,xshift=3em,yshift=-3em] {7: heartbeat (return task)} (jobtracker.south);

\path [dashed,black] ([xshift=0em] tasktracker.west) edge [trans] node[auto,black,fill =white,xshift=0em,yshift=0em] {8:retrieve job resources} (cloud);

\path [dashed,black] ([xshift=0em] tasktracker.south) edge [trans] node[auto,black,fill =blue!10,xshift=0.2em,yshift=-0.7em] {9:launch} (cjvm.north);

\path [dashed,black] ([xshift=0em] child.south) edge [trans] node[auto,black,fill =white,xshift=0em,yshift=-1.3em] {10:run} (maptask.north);

\end{tikzpicture}
\end{document}
